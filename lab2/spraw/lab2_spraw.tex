\documentclass[polish, 11pt]{article}

\usepackage[a4paper, margin=25mm]{geometry}
\usepackage{babel}
\usepackage{polski}
\usepackage[utf8]{inputenc}
\usepackage[T1]{fontenc}
\usepackage{lmodern}
\usepackage{listings}
\usepackage{xcolor}
\usepackage{gasasm}
\lstset{language=[GAS]Assembler}



\begin{document}
	\begin{flushright}
		Wrocław, 18 marca 2018
	\end{flushright}
	
	Marcin Kotas, 235098
	WT-P-7.15
	
	\begin{flushright}
		prowadząca: mgr Aleksandra Postawka
	\end{flushright}
	
	\begin{center}
		Laboratorium Architektury Komputerów
	
		(2)	Utrwalenie umiejętności tworzenia prostych konstrukcji programowych, obsługa plików
	\end{center}
	
	\section{Treść ćwiczenia}
	\subsection*{Zakres i program ćwiczenia:}
		Utworzenie programu wczytującego dwie liczby zapisane w kodzie szesnastkowym z plików
		i dodającego je z użyciem flagi przeniesienia.
		Wynik zapisywany w osobnym pliku w kodzie czwórkowym.
		
	\section{Kod programu}
	\subsection{Sekcja danych}
		
		\begin{minipage}{.5\textwidth}
			W sekcji danych umieszczone zostały dodatkowe nazwy symboliczne użyte do obsługi plików.
			Wartości \texttt{SYSOPEN} oraz \texttt{SYSCLOSE} odnoszą się do funkcji otwierania oraz zamykania plików.
			Wartość \texttt{O\_WR\_CRT\_TRNC} została wyliczona jako suma wartości dla tylko odczytu (01), tworzenia pliku jeśli nie ma (0100) oraz nadpisywania poprzednich wartości pliku (01000).
			Zmienne \texttt{f\_***} zawierają nazwy plików używanych w programie.

		\end{minipage}%
		\hspace{1cm}
		\begin{minipage}{.5\textwidth}	%cspell:disable
			\begin{lstlisting}	
.data
	SYSEXIT = 60
	EXIT_SUCCESS = 0
	SYSREAD = 0
	SYSWRITE = 1
	STDOUT = 1
	SYSOPEN = 2
	SYSCLOSE = 3
	O_RDONLY = 00
	O_WRONLY = 01
	O_WR_CRT_TRNC = 01101

	BUFLEN = 1024

	f_in1: .ascii "in1\0"
	f_in2: .ascii "in2\0"
	f_out: .ascii "out\0"

.bss
	.comm num1_buf, 1024
	.comm num1, 1024
	.comm num2_buf, 1024
	.comm num2, 1024
	.comm sum_buf, 1024
	.comm sum_out, 1024
			\end{lstlisting}	%cspell:enable
		\end{minipage}

	\subsection{Wczytywanie liczby z pliku}
	
		\begin{minipage}{.4\textwidth}
			Otwieranie pliku odbywa się za pomocą funkcji \texttt{SYSOPEN}.
			Pierwszym argumentem tej funkcji jest nazwa pliku.
			Drugim jest tryb otwarcia, a ostatnim kod dostępu.

			Następnie przy pomocy funkcji \texttt{SYSREAD} zawartość pliku wczytywana jest do bufora.

			Funkcja \texttt{SYSCLOSE} zamyka plik. Jej jedynym argumentem jest deskryptor pliku.

		\end{minipage}%
		\hspace{1cm}
		\begin{minipage}{.5\textwidth}	%cspell:disable
			\begin{lstlisting}	
num1_file:              # %r14 - number of num1 bytes
	movq $SYSOPEN, %rax
	movq $f_in1, %rdi
	movq $O_RDONLY, %rsi
	movq $0666, %rdx
	syscall

	movq %rax, %rdi  # file handle
	movq $SYSREAD, %rax
	movq $num1_buf, %rsi
	movq $BUFLEN, %rdx
	syscall

	movq %rax, %r14  
	sub $3, %r14       # -'\n', taking 2 bytes at a time

	movq $SYSCLOSE, %rax    # file handle still in %rdi
	syscall

			\end{lstlisting}	%cspell:enable
		\end{minipage}


	\subsection{Konwersja znaków w buforze na liczbę}
	
		\begin{minipage}{.4\textwidth}
			Pętla \texttt{read\_num1} sczytuje z bufora od końca po 2 bajtach,
			następnie sprawdza, czy oba bajty zawierają cyfrę w kodzie ascii.
			Służy do tego funkcja \texttt{to\_number},
			która sprawdza znak i zwraca cyfrę w rejestrze \texttt{al} oraz
			informację, czy znak był cyfrą w rejestrze \texttt{ah}.
			
			Następnie obie cyfry zapisywane są w jednym bajcie w buforze num1.
			Po sczytaniu wszystkich par bajtów następuje sprawdzenie,
			czy liczba bajtów była nieparzysta.
			Jeśli tak, następuje wczytanie bajtu o indeksie 0 i dopisanie go na koniec bufora.
		\end{minipage}%
		\hspace{1cm}
		\begin{minipage}{.5\textwidth}	%cspell:disable
			\begin{lstlisting}	
num1_decode:
	movq %r14, %r8
	movq %r8, %rdi
	movq $0, %rsi
	read_num1:
		movw num1_buf(, %rdi, 1), %bx
		movb %bl, %al
		call to_number
		cmp $0, %ah
		jne exit

		movb %al, %bl
		shl $4, %bl

		movb %bh, %al
		call to_number
		cmp $0, %ah
		jne exit

		or %al, %bl
		movb %bl, num1(, %rsi, 1)

		inc %rsi
		sub $2, %rdi
		cmp $0, %rdi
		jge read_num1
	movq %r8, %rax
	movq $0, %rdx
	movq $2, %r8
	div %r8
	cmp $0, %rdx
	je num2_decode

	movq $0, %rdi       # if odd number
	movb num1_buf(, %rdi, 1), %al
	call to_number
	cmp $0, %ah
	jne exit
	movb %al, num1(, %rsi, 1)

			\end{lstlisting}	%cspell:enable
		\end{minipage}

	\subsection{Dodanie liczb}

		\begin{minipage}{.4\textwidth}
			Najpierw dodawane jest bez przeniesienia ostatnie 16 cyfr liczb.
			Rejestr flag kopiowany jest na stos (\texttt{pushf}), ponieważ funkcja \texttt{cmp} nadpisuje flagę przeniesienia.
			Następnie w pętli \texttt{add\_carry} wynik dodawania kopiowany jest do bufora \texttt{sum\_buf}, rejestr flag zostaje przywrócony z kopca (\texttt{popf})
			i dodane jest z przeniesieniem kolejne 8 bajtów liczb.
			Po wykonaniu dodawania rejestr flagowy znów zostaje wrzucony na stos.
			Pętla wykonuje się dopóki wynik dodawania jest większy od zera.
		\end{minipage}%
		\hspace{1cm}
		\begin{minipage}{.5\textwidth}  %cspell:disable
			\begin{lstlisting}
add_no_carry:
	movq $0, %rdi
	movq num1(, %rdi, 8), %rax
	movq num2(, %rdi, 8), %rbx
	add %rbx, %rax
	pushf
	
	add_carry:
		movq %rax, sum_buf(, %rdi, 8)
		inc %rdi
		popf
		movq num1(, %rdi, 8), %rax
		movq num2(, %rdi, 8), %rbx
		adc %rbx, %rax
		pushf
		cmp $0, %rax
		jne add_carry
	popf
	adc $0, %rax        # add carry, if %rax=ff..f+1
	cmp $0, %rax
	je no_carry
	movb $1, sum_buf(, %rdi, 8)
	inc %rdi
			\end{lstlisting}  %cspell:enable
		\end{minipage}
	

	\subsection{Konwersja na kod czwórkowy}

		\begin{minipage}{.4\textwidth}
			Do rejestru \texttt{r11} zapisana jest liczba bajtów w buforze \texttt{sum\_buf}.
			Jest to indeks z pętli \texttt{add\_carry} pomnożony przez 8.

			W pętli \texttt{to\_stack} brane są kolejne 2 bity z bufora
			(poprzez wykonanie funkcji and z liczbą 3),
			do których dodawany jest kod ascii '0' i wrzucane są na stos.
			Następnie rejestr \texttt{al} przesuwany jest o 2 miejsca,
			aby mogły zostać pobrane kolejne 2 bity.

			Pętla \texttt{to\_quater} wykonuje się tak długo, aż odczytane zostaną wszystkie bajty z bufora (liczba bajtów w rejestrze \texttt{r11}).

			Po wykonaniu się pętli \texttt{to\_quater} następuje sczytanie kolejnych znaków z rejestru do bufora i wypisanie do pliku.
			Otwarcie pliku następuje przy użyciu tych samych funkcji co na początku,
			tym razem z użyciem parametru \texttt{O\_WR\_CRT\_TRNC} zamiast \texttt{O\_RDONLY}.

			Zapisanie bufora do pliku odbywa się za pomocą funkcji \texttt{SYSWRITE},
			gdzie jako pierwszy parametr podawany jest deskryptor otwartego pliku.

		\end{minipage}%
		\hspace{1cm}
		\begin{minipage}{.5\textwidth}  %cspell:disable
			\begin{lstlisting}
no_carry:
	movq %rdi, %rax
	movq $8, %r8
	mul %r8
	movq %rax, %r11         # %r11 - number of sum_buf bytes
	movq $0, %rdi
	movq $0, %r9            # to_stack iterator
	to_quater:
		movb sum_buf(, %rdi, 1), %al
		inc %rdi
		to_stack:
			movb %al, %bl
			and $3, %bl
			shr $2, %al
			add $'0', %bl
			push %rbx
			inc %r9
			cmp $4, %r9
			jl to_stack

		movq $0, %r9
		cmp %r11, %rdi
		jl to_quater
			\end{lstlisting}  %cspell:enable
		\end{minipage}

	\section{Wnioski}
		Program uruchomił się poprawnie.
		Przy otwieraniu plików można podać jako parametr kod dostępu 0666 (ósemkowo),
		który nadaje wszystkim użytkownikom pełną kontrolę nad plikiem.
		Przy zapisywaniu liczb do pamięci należy pamiętać o konwencji little endian,
		gdzie liczby należy zapisywać od najmłodszego bitu.
\end{document}