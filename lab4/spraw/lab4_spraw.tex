\documentclass[polish, 11pt]{article}

\usepackage[a4paper, margin=25mm]{geometry}
\usepackage{babel}
\usepackage{polski}
\usepackage[utf8]{inputenc}
\usepackage[T1]{fontenc}
\usepackage{lmodern}
\usepackage{listings}
\usepackage{xcolor}
\usepackage{gasasm}
\lstset{language=[GAS]Assembler}


\begin{document}
	\begin{flushright}
		Wrocław, \today
	\end{flushright}
	
	Marcin Kotas, 235098
	WT-P-7.15
	
	\begin{flushright}
		prowadząca: mgr Aleksandra Postawka
	\end{flushright}
	
	\begin{center}
		Laboratorium Architektury Komputerów
	
		(4)	Łączenie różnych języków programowania w jednym projekcie
	\end{center}
	
	\vspace{8cm}

\section{Treść ćwiczenia}
	\subsection*{Zakres i program ćwiczenia:}
		Pierwszy program napisany w języku asemblera wywołuje funkcje napisane w języku C.
		Wczytanie argumentu całkowitoliczbowego i zmiennoprzecinkowego za pomocą funkcji bibliotecznej scanf,
		wywołanie funkcji napisanej w języku C:
		\begin{lstlisting}[language=C,frame=none,gobble=12]
			double f(float x, int y);
			   f(x, y) = x^2 + y^3
		\end{lstlisting}
		Dwukrotne wypisanie wyniku za pomocą pojedynczego wywołania funkcji printf \verb|"%lf \n %lf"|.

		Drugi program napisany w języku C zawiera wstawkę w języku asemblera,
		obliczającą wartość liczby danej za pomocą łańcucha znaków w reprezentacji siódemkowej,
		np. ''14352'' i zapisanie wyniku w zmiennej globalnej w jęz. C. Można również przekazać długość.
		
		Trzeci program napisany w języku C wywołuje funkcje napisane w języku asemblera.
		Funkcja obliczająca wartość liczby danej w zapisie tekstowym w reprezentacji,
		w której wagami pozycji są kolejne wartości silni (mniejsze wagi dla bardziej znaczących cyfr).

\newpage
\section{Program pierwszy}
	\subsection{Sekcja danych}
		\begin{minipage}{.5\textwidth}
			W sekcji danych umieszczone zostały zmienne definiujące argumenty funkcji \verb|scanf| oraz \verb|printf|.			
		\end{minipage}%
		\hspace{1cm}
		\begin{minipage}{.5\textwidth}
			\lstinputlisting[firstline=10,lastline=15]{../prg1.s}
		\end{minipage}

	\subsection{Wywoływanie zewnętrznych funkcji}	
		\begin{minipage}{.5\textwidth}
			Do rejestru \verb|rax| podawana jest liczba argumentów zmiennoprzecinkowych przekazywanych do funkcji.
			W rejestrach \verb|rdi| i \verb|rsi| znajdują się pierwsze dwa argumenty wywoływanej funkcji, zgodnie z konwencją wywoływania.
			Pierwsze wywołanie funkcji wczytuje liczbę zmiennoprzecinkową, a drugie liczbę całkowitą.

			Podczas wywoływania funkcji z argumentami zmiennoprzecinkowymi przekazywane są one przez rejestry \verb|xmm[0-7]|.
			
			Funkcja w C ma postać

			Wartość zmiennoprzecinkowa zwrócona przez funkcję znajduje się w rejestrze \verb|xmm0|.
			Wartość ta jest kopiowana do rejestru \verb|xmm1|, aby mogła zostać przekazana do \verb|printf| dwa razy.

			Zgodnie z \verb|System V AMD64 ABI| (Interfejs binarny aplikacji 64 bitowych wykorzystywany m.in. przez Linux)
			stos przed wywołaniem funkcji powinien być zawsze wyrównany do 16 bajtów (adres stosu wielokrotnością 16).
			Z tego powodu przed wywołaniem funkcji \verb|printf| następuje odjęcie 8 od \verb|rsp|
			(po uruchomieniu programu stos przesunięty jest o 8 bajtów).

			Po wykonaniu funkcji do stosu dodawane jest 8, aby przywrócić go do pierwotnego stanu.

		\end{minipage}%
		\hspace{1cm}
		\begin{minipage}{.5\textwidth}
			\lstinputlisting[firstline=20,lastline=44]{../prg1.s}
		\end{minipage}

\section{Program drugi}
	Kod w języku asemblera wstawiany jest w następującym formacie:

	\begin{lstlisting}[language=C,frame=none,gobble=8]
		asm("instrukcje asemblerowe" 
			: zwracane wartosci //opcjonalne
			: argumenty			//
			: uzywane rejestry  //
			);			
	\end{lstlisting}

	\subsection{Instrukcje}
		W instrukcjach asemblera do rejestrów należy odwoływać się z dwoma znakami \verb|%%|.
		Do argumentów można odwoływać się za pomocą \verb|%n|, gdzie \verb|n| jest numerem argumentu.
		Rejestry numerowane są od 0 począwszy od wartości zwracanych
		(jeśli funkcja zwraca jedną wartość i przyjmuje dwa argumenty, to argument pierwszy będzie miał numer 1 itd.).
		Ten sposób odwoływania się do rejestrów jest potrzebny,
		jeśli nie sprecyzuje się, w których rejestrach mają być przechowywane wartości.

	\subsection{Używane rejestry}
		Zmienne przypisuje się do rejestrów w następującym formacie: \verb|': "<rej>"(<zm>)'|, gdzie:

		\verb|<rej>| może być \verb|'r'|, aby kompilator sam wybrał używany rejestr,
		lub symbolem rejestru (np. \verb|'a'|, aby użyć rejestru \verb|rax|).

		Przed symbolem rejestru może być dodany modyfikator \verb|'='|,
		który oznacza, że zmienna użyta będzie tylko do zapisu (zwracanie wyniku).

		\verb|<zm>| to zmienna zadeklarowana w programie.
		Jeżeli we wstawce asemblerowej używa się pełnych, 64 bitowych rejestrów,
		zmienna również powinna być 64 bitowa (typ long).

	\subsection{Kod}
		\begin{minipage}{.45\textwidth}
			Wstawka asemblerowa traktuje ciąg znaków pod adresem \verb|text| jako liczbę w reprezentacji siódemkowej
			i oblicza jej wartość dziesiętną.
			
			Wynik przekazywany jest z rejestru \verb|rax| do zmiennej \verb|res|.

			Adres ciągu znaków \verb|text| przekazywany jest do rejestru \verb|rdi|,
			a długość ciągu \verb|length| do rejestru \verb|rsi|.

			Funkcje asemblerowe odwołują się do argumentów bezpośrednio przez zadeklarowane rejestry.

		\end{minipage}%
		\hspace{1cm}
		\begin{minipage}{.5\textwidth}
			\lstinputlisting[firstline=10,lastline=35,language=C,framexleftmargin=-16pt,xleftmargin=-20pt]{../prg2.c}
		\end{minipage}

\newpage
\section{Program trzeci}
	Program wywołuje funkcję napisaną w jęz.\ asemblera, która oblicza wartość liczby danej w zapisie tekstowym w reprezentacji,
	w której wagami pozycji są kolejne wartości silni (mniejsze wagi dla bardziej znaczących cyfr).
	
	\subsection{Program główny}
		\begin{minipage}{.5\textwidth}
			Na początku programu zadeklarowana zostaje funkcja zewnętrzna za pomocą słowa kluczowego \verb|extern|.
			Podane muszą zostać typ zwracany przez funkcję, jej nazwa oraz typy argumentów przyjmowanych przez funkcję.

		\end{minipage}%
		\hspace{1cm}
		\begin{minipage}{.5\textwidth}
			\lstinputlisting[firstline=9,lastline=17,language=C]{../prg3.c}
		\end{minipage}
	
	\subsection{Funkcja w jęz.\ asemblera}
		\begin{minipage}{.5\textwidth}
			Funkcja musi zostać zadeklarowana globalnie, co odbywa się na samym początku kodu.

			Argumenty przekazane do funkcji znajdują się w \verb|rdi(text)| oraz \verb|rsi(length)|.
			
			Funkcja pobiera kolejne znaki z tekstu, a następnie wymnaża przez aktualną wartość silni, począwszy od 1.
			Po wymnożeniu wartość dodawana jest do sumy w \verb|rbx|,
			a wartość silni wymnażana jest przez wartość w rejestrze \verb|r9|, który jest po tej operacji inkrementowany.

			Pętla wykonuje się dopóki licznik w \verb|rcx| nie przekroczy długości tekstu w \verb|rsi|.

			Na końcu suma z \verb|rbx| kopiowana jest do \verb|rax|, gdyż w tym rejestrze zwracany jest wynik funkcji.
			Jeżeli na którejś pozycji w tekście wystąpił znak inny niż cyfra, zwracany jest wynik \verb|-1|.
			
			\section{Wnioski}
				Wszystkie programy uruchomiły się poprawnie.
			
				W programie drugim zaszła potrzeba wyrównania stosu do 16 bajtów podczas wywoływania funkcji \verb|printf| z argumentami zmiennoprzecinkowymi.
				Mimo iż zgodnie z \verb|System V AMD64 ABI| stos musi zawsze być wyrównany przed wywoływaniem jakiejkolwiek funkcji, często nie powoduje to błędu.
		\end{minipage}%
		\hspace{1cm}
		\begin{minipage}{.5\textwidth}
			\lstinputlisting[firstline=1,lastline=31]{../prg3.s}
		\end{minipage}

\end{document}