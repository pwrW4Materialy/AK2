\documentclass[polish, 11pt]{article}

\usepackage[a4paper, margin=25mm]{geometry}
\usepackage{babel}
\usepackage{polski}
\usepackage[utf8]{inputenc}
\usepackage[T1]{fontenc}
\usepackage{lmodern}
\usepackage{listings}
\usepackage{xcolor}
\usepackage{gasasm}
\lstset{language=[GAS]Assembler}


\begin{document}
	\begin{flushright}
		Wrocław, 3 marca 2018
	\end{flushright}
	
	Marcin Kotas, 235098
	WT-P-7.15
	
	\begin{flushright}
	prowadząca: Aleksandra Postawka
	\end{flushright}
	
	\begin{center}
	Laboratorium Architektury Komputerów
	
	(0) Podstawy uruchamiania programów asemblerowych na platformie Linux
	\end{center}
	
	\section{Treść ćwiczenia}
	\subsection*{Zakres i program ćwiczenia:}
		Techniki tworzenia i uruchamiania programów napisanych w języku asemblera AT\&T. Obsługa debugger'a gdb.
		
	\subsection*{Zrealizowane zadania:}
		Utworzenie prostego programu ,,Hello, world'' oraz pliku Makefile w celu zautomatyzowania kompilacji
		i konsolidacji programu za pomocą komendy make.
		
		Napisanie programu wczytującego tekst ze standardowego wejścia,
		zamieniającego wielkie litery na małe i na odwrót.
		
	\section{Przebieg ćwiczenia}
	\subsection{Program ,,Hello, world!''}
		
		\begin{minipage}{.5\textwidth}
		Na początku programu znajduje się sekcja danych (.data).
		Zawiera ona definicje nazw symbolicznych oraz deklaracje zmiennych i buforów.
		
		Sekcja .text zawiera zapis algorytmu programu. \_start oznacza wejście.
		Funkcja movq kopiuje do kolejnych rejestrów nazwę a następnie argumenty funkcji.
		Dyrektywa syscall wywołuje funkcję. Funkcja SYSWRITE służy do wydrukowania napisu,
		a funkcja SYSEXIT wychodzi z programu.
		\end{minipage}%
		\hspace{1cm}
		\begin{minipage}{.5\textwidth}
			\begin{lstlisting}
.data
	SYSREAD = 0
	SYSWRITE = 1
	SYSEXIT = 60
	STDOUT = 1
	STDIN = 0
	EXIT_SUCCESS = 0

	buf: .ascii "Hello, world!\n"
	buf_len= .-buf

.text
.globl _start
_start:
	movq $SYSWRITE, %rax
	movq $STDOUT, %rdi
	movq $buf, %rsi
	movq $buf_len, %rdx
	syscall
	
	movq $SYSEXIT, %rax
	movq $EXIT_SUCCESS, %rdi
	syscall
			\end{lstlisting}
		\end{minipage}
		
	\subsection{Kompilacja programu}
		Plik należy najpierw skompilować do obiektu za pomocą funkcji \texttt{as -o plik.o plik.s}.
		Następnie plik.o należy skonsolidować przy użyciu funkcji \texttt{ld -o plik plik.o}.
		W celu zapamiętania konfiguracji kompilacji warto stworzyć plik Makefile,
		w którym zachowane będą kolejne instrukcje:
		\begin{lstlisting}[tabsize=5]
plik:	plik.o
		ld -o plik plik.o
plik.o:	plik.s
		as -o plik.o plik.s
		\end{lstlisting}
		Aby wykonać kompilację za pomocą tego pliku należy użyć w terminalu komendy \texttt{make}.
		Aby uruchomić program należy użyć komendy \texttt{./plik} będąc w folderze z programem.
		
	\subsection{Debugowanie programu}
		Aby przeprowadzić debugowanie programu za pomocą gdb należy skompilować program z odpowiednimi opcjami,
		które powiążą polecenia i etykiety z końcowymi kodami.
		Można to zrobić kompilująć program za pomocą kompilatora gcc: \texttt{gcc -g -o plik plik.s},
		przy czym należy pamiętać, by w kodzie programu zamienić \texttt{.globl} na \texttt{.global}
		oraz \texttt{\_start} na \texttt{main}.
		
		Drugim sposobem na kompilację programu z obsługą debugowania jest dodanie flagi \texttt{-gstabs}
		przy wywoływaniu kompilatora \texttt{as}. Nie trzeba wtedy zmieniać nazw w kodzie,
		jednak w starszych wersjach debugera istniała potrzeba dodania funkcji \texttt{nop}
		na początek programu ponieważ nie dało się ustawić breakpoint'a na etykiecie \_start.
		
		Gdb umożliwia dokładne śledzenie działania programu poprzez zatrzymywanie wykonywania w konkretnym miejscu,
		a następnie śledzenie krok po kroku i monitorowanie zawartości rejestrów.
		Aby ustawić breakpoint na konkretnej instrukcji należy wpisać komendę \texttt{b [adres]}.
		Można posłużyć się etykietami: \texttt{b *\_start}. Do uruchomienia programu służy komenda \texttt{r}.
		Po zatrzymaniu programu można wykonywać instrukcje krok po kroku (\texttt{s})
		lub kontynuować normalną pracę programu (\texttt{c}).
		W każdej chwili pożna wyświetlić zawartość wszystkich rejestrów (\texttt{info registers})
		lub poszczególnych (\texttt{p/d \$<nazwa rejestru>}).
		Można również wyświetlić zawartość zmiennych odnosząc się do nich za pomocą \texttt{\&<nazwa zmiennej>}.
	
	\subsection{Program zamieniający wielkość liter}
		\begin{minipage}{.5\textwidth}
		W sekcji data dodana została zmienna BUFLEN, która definiuje długość w bajtach bufora,
		do którego wczytywany będzie tekst.
		
		Sekcja .bss zawiera niezainicjalizowane dane.
		Tam znajdują się bufory, do których zostanie zapisany wprowadzony tekst.
		\end{minipage}%
		\hspace{1cm}
		\begin{minipage}{.5\textwidth}
			\begin{lstlisting}
.data
    SYSREAD = 0
    SYSWRITE = 1
    SYSEXIT = 60
    STDOUT = 1
    STDIN = 0
    EXIT_SUCCESS = 0
    BUFLEN = 512

.bss
    .comm buf_in, 512
    .comm buf_out, 512
			\end{lstlisting}
		\end{minipage}
		
		\begin{minipage}{.4\textwidth}
		Do wczytania tekstu z klawiatury wykorzystana jest funkcja \texttt{SYSREAD},
		a jako argument podany jest standardowy strumień wejściowy \texttt{STDIN}.
		Wprowadzony tekst zapisywany jest do bufora \texttt{buf\_in}.
		Następnie zawartość rejestru \texttt{rax} jest zmniejszana o 1, aby uwzględnić znak końca linii.  
		
		\vspace{1cm}
		W pętli \texttt{zamien\_litery} zawartość bufora jest kopiowana po jednym bajcie do rejestru \texttt{bh}.
		Poprzez porównanie z kodami liter za pomocą funkcji \texttt{cmp},
		znak w rejestrze klasyfikowany jest jako litera lub inny znak.		
		Jeżeli znak jest literą, to poprzez funkcję \texttt{xor} zamieniana jest wielkość litery (liczba \texttt{0x20}
		posiada tylko jeden bit 1 na miejscu, w którym różnią się liczby wielkie od małych).
		Na końcu pętli aktualny znak jest kopiowany do bufora wyjściowego,
		po czym porównywana jest wartość licznika (rejestr \texttt{rdi}) z ilością znaków w buforze (rejestr \texttt{rax}).
		W tym przypadku wykorzystywany jest skok warunkowy \texttt{jl} - jump if less.
		
		\vspace{1cm}
		Po wykonaniu się pętli do bufora dodawany jest znak końca linii,
		po czym funkcja SYSWRITE drukuje do strumienia \texttt{STDOUT} zawartość bufora \texttt{buf\_out}.
		Na końcu wykonuje się funkcja SYSEXIT.
		\end{minipage}%
		\hspace{1cm}j
		\begin{minipage}{.5\textwidth}
			\begin{lstlisting}
.text
.globl _start

_start:
    movq $SYSREAD, %rax
    movq $STDIN, %rdi
    movq $buf_in, %rsi
    movq $BUFLEN, %rdx
    syscall

    dec %rax	#'\n'
    movq $0, %rdi	#licznik

    zamien_litery:
        movb buf_in(,%rdi, 1), %bh
        cmp $'A', %bh
        jl nie_litera

        cmp $'Z', %bh
        jle litera

        cmp $'a', %bh
        jl nie_litera

        cmp $'z', %bh
        jg nie_litera

        litera:
        movb $0x20, %bl
        xor %bl, %bh

        nie_litera:
        movb %bh, buf_out(, %rdi, 1)
        inc %rdi
        cmp %rax, %rdi
        jl zamien_litery

    movb $'\n', buf_out(, %rdi, 1)

    movq $SYSWRITE, %rax
    movq $STDOUT, %rdi
    movq $buf_out, %rsi
    movq $BUFLEN, %rdx
    syscall

    movq $SYSEXIT, %rax
    movq $EXIT_SUCCESS, %rdi
    syscall
			\end{lstlisting}
		\end{minipage}
		
	\section{Wnioski}
		Programy uruchomiły się poprawnie.
		W nowszych wersjach gdb nie ma już potrzeby wstawiania funkcji no operation po \_start,
		jeśli używa się kompilatora asemblerowego (as).
		Można również kompilować używająć kompilatora c (gcc), który sprawdza się równie dobrze.
\end{document}