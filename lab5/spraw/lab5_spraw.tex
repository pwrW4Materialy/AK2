\documentclass[polish, 11pt]{article}

\usepackage[a4paper, margin=25mm]{geometry}
\usepackage{babel}
\usepackage{polski}
\usepackage[utf8]{inputenc}
\usepackage[T1]{fontenc}
\usepackage{lmodern}
\usepackage{listings}
\usepackage{xcolor}
\usepackage{gasasm}
\lstset{language=[GAS]Assembler}


\begin{document}
	\begin{flushright}
		Wrocław, \today
	\end{flushright}
	
	Marcin Kotas, 235098
	WT-P-7.15
	
	\begin{flushright}
		prowadząca: mgr Aleksandra Postawka
	\end{flushright}
	
	\begin{center}
		Laboratorium Architektury Komputerów
	
		(5)	Jednostka zmiennoprzecinkowa (FPU)
	\end{center}
	
	\vspace{8cm}

\section{Treść ćwiczenia}
	\subsection*{Zakres i program ćwiczenia:}
		Pierwszy program napisany w języku C wywołuje funkcje napisane w języku asemblera:
		\begin{enumerate}
			\item pozwalająca na sprawdzenie wystąpienia wyjątków,
			\item pozwalająca na ustawienie wybranego bitu maski wyjątków
		\end{enumerate}

		Drugi program napisany w języku C wywołuje funkcję napisaną w języku asemblera:\newline
		aproksymacja funkcji ln(1+x) za pomocą szeregu Taylora

		x - argument zmiennoprzecinkowy
		
		n - liczba iteracji

\newpage
\section{Program pierwszy}
	\subsection{Program główny}
			W programie dostępne jest menu zaimplementowane jako pętla.
			Do wyboru są następujące operacje:
			\begin{enumerate}
				\item Sprawdzanie, czy wystąpiły wyjątki,
				\item Ustawienie wybranych masek wyjątków,
				\item Wyzerowanie wszystkich masek,
				\item Przeprowadzenie testu dzielenia przez zero
			\end{enumerate}

			Pierwsza opcja wywołuje funkcję zewnętrzną \verb|checkExc()|,
			która zwraca liczbę definiującą znalezione wyjątki.
			Następnie po kolei sprawdza flagi poprzez wykonanie operacji modulo 2 (sprawdzenie czy bit zerowy jest ustawiony)
			i przesunięcie bitowe w prawo aby sprawdzić kolejną flagę.

			Druga opcja pozwala na wybranie maski do ustawienia --- od 0 do 5.
			Następnie wywoływana jest zewnętrzna funkcja \verb|maskExc(int exc)|.
			Jako argument przekazywana jest liczba 2 podniesiona do potęgi równej wybranej masce.
			W ten sposób uzyskiwany jest ustawiony bit na pozycji odpowiadającej wybranej masce.

			Opcje trzecia oraz czwarta wywołują odpowiednio \verb|clrMasks()| oraz \verb|testDivByZero()|.
			
			\lstinputlisting[firstline=12,lastline=71,framexleftmargin=-50pt,xleftmargin=-60pt]{../prg1.c}

	\subsection{Funkcje asemblerowe}	
		\begin{minipage}{.5\textwidth}
			Wszystkie funkcje są zadeklarowane w obrębie jednego pliku.
		\end{minipage}%
		\hspace{1cm}
		\begin{minipage}{.5\textwidth}
			\lstinputlisting[firstline=4,lastline=9]{../prg1.s}
		\end{minipage}

		\subsubsection{checkExc}
			\begin{minipage}{.5\textwidth}
				Funkcja wczytuje zawartość rejestru statusu do rejestru \verb|ax|
				za pomocą funkcji \verb|fstsw|.
				Funkcja \verb|fwait| wykonywana jest, aby upewnić się,
				że następna instrukcja wykona się po wczytaniu rejestru statusu, a nie w trakcie.
				Następnie czyści starszą część rejestru, aby zostawić tylko flagi wyjątków.
				Wynik zwraca przez rejestr \verb|rax|.
			\end{minipage}%
			\hspace{1cm}
			\begin{minipage}{.5\textwidth}
				\lstinputlisting[firstline=10,lastline=15]{../prg1.s}
			\end{minipage}

		\subsubsection{clrMasks}
			\begin{minipage}{.5\textwidth}
				Funkcja wczytuje zawartość rejestru kontrolnego do pamięci
				za pomocą funkcji \verb|fnstcw|, a następnie kopiuje do rejestru \verb|ax|.
				Następnie czyści młodsze 6 bitów rejestru, które odpowiadają maskom wyjątków.
				Na koniec ładuje zawartość rejestru \verb|rax| do pamięci
				i ładuje do rejestru kontrolnego za pomocą funkcji \verb|fldcw|.
			\end{minipage}%
			\hspace{1cm}
			\begin{minipage}{.5\textwidth}
				\lstinputlisting[firstline=18,lastline=25]{../prg1.s}
			\end{minipage}

		\subsubsection{maskExc}
			\begin{minipage}{.5\textwidth}
				Funkcja wczytuje zawartość rejestru kontrolnego jak poprzednio.
				Następnie operacją \verb|OR| ustawia odpowiednie maski przekazane do funkcji w \verb|rdi|.
				Na koniec ładuje zawartość rejestru \verb|rax| do rejestru kontrolnego.
			\end{minipage}%
			\hspace{1cm}
			\begin{minipage}{.5\textwidth}
				\lstinputlisting[firstline=27,lastline=34]{../prg1.s}
			\end{minipage}

		\subsubsection{testDivByZero}
			\begin{minipage}{.5\textwidth}
				Na początku ładowane są do stosu FPU zero oraz jeden.
				Następnie wykonywane jest dzielenie \verb|ST(0) / ST(1)|
				funkcją \verb|fdivp|, która również usunie wartość na szczycie stosu.
				Na koniec usuwana jest ostatnia wartość pozostająca w stosie.
			\end{minipage}%
			\hspace{1cm}
			\begin{minipage}{.5\textwidth}
				\lstinputlisting[firstline=36,lastline=41]{../prg1.s}
			\end{minipage}

\section{Program drugi}
	\subsection{Program główny}
		Program wczytuje argumenty zewnętrznej funkcji \verb|lnApprox| za pomocą funkcji \verb|printf|.
		\verb|x| musi zawierać się w przedziale \((-1,1]\), ponieważ rozszerzenie Taylora bazuje na sumie szeregu geometrycznego.
		\verb|steps| oznacza ilość kroków do wykonania.
		Następnie drukuje wynik wywołania tej funkcji.

		\lstinputlisting[firstline=6,lastline=14,framexleftmargin=-16pt,xleftmargin=-20pt]{../prg2.c}

	\subsection{Funkcja lnApprox}
		Kolejność działań bazuje na następującym wzorze:

		\begin{displaymath}
			\ln(1+x)=x-\frac{x^2}{2}+\frac{x^3}{3}-\frac{x^4}{4}+\cdots	
		\end{displaymath}

		\subsubsection{Ładowanie zmiennych do FPU}
			Na początku funkcja przekazuje argument z \verb|xmm0| na stos, aby możliwe było załadowanie go do rejestru FPU.\ 
			Następnie potrzebne zmienne i stałe ładowane są do \verb|FPU|: 
			\begin{itemize}
				\item \verb|finit| czyści i inicjalizuje jednostkę \verb|FPU|
				\item \verb|fld1| ładuje do rejestru \verb|FPU| wartość \(1\)
				\item \verb|fldl (%rsp)| wczytuje liczbę ze stosu
				\item \verb|fldz| ładuje wartość \(0\)
			\end{itemize}

			\lstinputlisting[firstline=9,lastline=17]{../prg2.s}

		\subsubsection{Pętla ln\_loop}			
			W rejestrze \verb|ST(0)| znajdują się kolejne potęgi \verb|x|.
			Przed uruchomieniem pętli jego wartość wynosi 1.
			Na początku każdej iteracji pętli wartość tego rejestru wymnażana jest przez \verb|x| (\verb|ST(2)|).
			Następnie wartość tego rejestru jest ładowana ponownie (\verb|fld %st|), aby można było wykonać na niej działania
			jednocześnie zachowując potęgę \verb|x| do dalszych obliczeń.

			Funkcja \verb|fdiv| dokonuje dzielenia \verb|ST(0)/ST(4)|.
			\verb|faddp| dodaje wartość \verb|ST(0)| do \verb|ST(2)| oraz usuwa wartość z \verb|ST(0)|.

			Następnie na stos FPU ładowana jest wartość 1, która dodawana jest do \verb|ST(4)|, w celu inkrementacji dzielnika.
			Funkcja \verb|fchs| zmienia znak \verb|ST(0)| (kolejne potęgi \verb|x|) na przeciwny.

			\lstinputlisting[firstline=19,lastline=30]{../prg2.s}

		\subsubsection{Zwracanie wartości}
			Funkcja najpierw usuwa wartość z góry (potęgi dwójki) za pomocą \verb|fstp %st|
			--- kopiowanie na siebie i `pop'.
			Następnie przenosi wynik ze stosu FPU na wskaźnik stosu --- \verb|fstpl (%rsp)|,
			po czym kopiuje tą wartość do rejestru \verb|xmm0|.

			Na końcu funkcja usuwa dwie pozostałe wartości ze stosu FPU.\ 

			\lstinputlisting[firstline=32,lastline=37]{../prg2.s}

\section{Wnioski}
	Podczas korzystania z jednostki FPU należy pamiętać o `przesuwaniu' się rejestrów podczas ładowania wartości.
	Przed rozpoczęciem działań warto wywołać funkcję \verb|finit|, aby upewnić się, że stos FPU jest pusty oraz ustawienia są domyślne.
	Podczas ładowania oraz odkładania wartości z pamięci lub stosu należy pamiętać o odpowiednim przyrostku:
	\begin{itemize}
		\item \verb|s| jeśli wartość jest 32 bitowa (single)
		\item \verb|l| jeśli jest 64 bitowa (double)
		\item \verb|t| jeśli jest 80 bitowa (extended double)
	\end{itemize}

	Przyrostki te nie są zgodne z tymi stosowanymi w każdym innym miejscu w języku asemblera
	(\verb|l| dla wartości 32 bitowych, \verb|q| dla 64 bitowych).
	Odnoszą się one do nazw formatów zmiennoprzecinkowych --- \verb|s|hort real, \verb|l|ong real, \verb|t|emporary real.
\end{document}