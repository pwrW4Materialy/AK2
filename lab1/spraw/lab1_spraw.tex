\documentclass[polish, 11pt]{article}

\usepackage[a4paper, margin=25mm]{geometry}
\usepackage{babel}
\usepackage{polski}
\usepackage[utf8]{inputenc}
\usepackage[T1]{fontenc}
\usepackage{lmodern}
\usepackage{listings}
\usepackage{xcolor}
\usepackage{gasasm}
\lstset{language=[GAS]Assembler}



\begin{document}
	\begin{flushright}
		Wrocław, 11 marca 2018
	\end{flushright}
	
	Marcin Kotas, 235098
	WT-P-7.15
	
	\begin{flushright}
		prowadząca: mgr Aleksandra Postawka
	\end{flushright}
	
	\begin{center}
		Laboratorium Architektury Komputerów
	
		(1)	Utrwalenie umiejętności konstruowania programów wykorzystujących pętle i sprawdzanie warunków.
	\end{center}
	
	\section{Treść ćwiczenia}
	\subsection*{Zakres i program ćwiczenia:}
		Utworzenie programu wczytującego liczbę ze standardowego strumienia
		i drukującego dolne przybliżenie pierwiastka kwadratowego z tej liczby.
		
	\section{Kod programu}
	\subsection{Zapisanie liczby z bufora do rejestru}
		
		\begin{minipage}{.4\textwidth}
			W pętli to\_number wykonuje się bajt po bajcie sprawdzanie, czy znak jest kodem cyfry.
			Jeśli tak, to odejmowany jest kod znaku ascii '0',
			po czym cyfra dodawana jest do całej liczby (znajdującej się w rejestrze rax)
			przemnożonej przez 10.
			W ten sposób liczba w rejestrze odpowiada liczbie wprowadzonej na klawiaturze.

		\end{minipage}%
		\hspace{1cm}
		\begin{minipage}{.5\textwidth}	%cspell:disable
			\begin{lstlisting}	
    movq %rax, %r8
    dec %r8        	# -'\n'
    movq $0, %rbx	
    movq $0, %rdi	# iterator
    movq $10, %r10	# podstawa
    movq $0, %rax	# liczba 

    to_number:
        movb textin(, %rdi, 1), %bl
        cmp $'0', %bl
        jl not_number
        cmp $'9', %bl
        jg not_number

        sub $'0', %bl

        mul %r10
        add %rbx, %rax

        inc %rdi
        cmp %r8, %rdi
        jl to_number
			\end{lstlisting}	%cspell:enable
		\end{minipage}

	\subsection{Obliczenie pierwiastka}
	
		\begin{minipage}{.4\textwidth}
			Następnie w pętli square\_root obliczana jest suma kolejnych liczb nieparzystych.
			W momencie, kiedy suma przekroczy wartość wprowadzonej liczby,
			ilość dodanych liczb nieparzystych wyznacza górne oszacowanie pierwiastka.
			Aby otrzymać dolne oszacowanie wartość ta jest dekrementowana.

			Pętla to\_stack wyłuskuje z wyniku kolejne cyfry dzieląc przez 10,
			po czym dodaje do nich kod znaku '0' oraz wrzuca na stos. 
			Na koniec w pętli to\_text przenosi ze stosu cyfry wyniku do bufora textout,
			aż do uzyskania pełnego wyniku.

		\end{minipage}%
		\hspace{1cm}
		\begin{minipage}{.5\textwidth}	%cspell:disable
			\begin{lstlisting}	
	movq $0, %r8	# pierwiastek
	movq $1, %rdi   # kolejne l. nieparzyste
	movq $0, %r12   # suma -||-

	square_root:
		add %rdi, %r12
		add $2, %rdi
		inc %r8
		cmp %rax, %r12
		jle square_root

	dec %r8	# dolne oszacowanie

	movq %r8, %rax
	movq $0, %r8
	to_stack:
		div %r10
		add $'0', %rdx
		push %rdx
		movq $0, %rdx
		inc %r8
		cmp $0, %rax
		jg to_stack

	movq $0, %rdi
	to_text:
		pop textout(, %rdi, 1)
		inc %rdi
		cmp %r8, %rdi
		jl to_text

	movb $'\n', textout(, %rdi, 1)
			\end{lstlisting}	%cspell:enable
		\end{minipage}

	\section{Wnioski}
		Program uruchomił się poprawnie.
		Pętla zamieniająca liczbę na znaki ascii pobiera cyfry od końca,
		przez co w buforze cyfra zapisana jest od prawej do lewej.
		Aby zamienić kolejność cyfr wykorzystany został stos.
\end{document}